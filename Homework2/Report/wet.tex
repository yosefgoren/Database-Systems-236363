\part{Wet}
\section{Entities}
The entities relevant to our DB are:
\begin{itemize}
	\item Photos: \texttt{ptable(photoID, description, size)}
	\item Disks: \texttt{dtable(diskID, company, speed, free\_space, cost)}
	\item RAMs: \texttt{rtable(ramID, company, size)}
\end{itemize}

\section{Relations}
The relations that we will store directly in our DB are:
\begin{itemize}
	\item Photo On Disk: \texttt{(podtable(photoID, diskID))}
	\item RAM On Disk: \texttt{(rodtable(ramID, diskID))}
\end{itemize}

\section{Relations \& Properties}
Note that for each of the Entity tables, the ID is used
as a primary key, and that each of these ID's are used
as foreign keys in the tables that link the Disks table to both
the RAMs table and the Photos table.\\
Also, we not that we have a 'many to one' relation in two cases:
\begin{enumerate}
	\item $photoID \longrightarrow diskID$ (in the \texttt{podtable})
	\item $ramID \longrightarrow diskID$ (in the \texttt{rodtable})
\end{enumerate}

\section{ERD Design Considirations}
Our main considirations upon choosing this design were
it's ability to be very space efficient, since there are
no duplicates in a non-key field and there are also no
potential NULL attributes allowed in our table.\\
This design also made sense to us due to
the requirement of the CURD API - to which this design
enabled a elegan implementation.\\
Moveover - with this implementation we were also
capable of implementing the more advanced APIs efficiently.

\section{API Implementation Design Considirations}
\subsection{CRUD API}
Most of the CRUD API functions were simple Getters \& Setters
and hence did not require much in terms of design; yet,
we were able to simplify the implementation of the CRUD API
by having implicit deletes thanks to proper usage of CASCDE
semantics in the definition of our tables.\\
Additionally, some of the methods required more sophisticated
implementations; noteably \texttt{deletePhoto()} had a more complex implementation
since we had to be able to recognize when
an update shouldnt be made AND make an update if nessecery - all in one querry.

\subsection{Basic API}
In the Basic API there were a few methods with intresting
design constrains:
\begin{itemize}
	\item \texttt{removePhotoFromDisk()}:
		In this method - the main issue was we need to updated
		the free space of the associated disk properly - even if the
		photo did not actually exist. While the trivial implementation of
		this method is quite simple - this edge case required us to find
		a much more complicated implementation - in which we
		relay on the properties of the 'SUM' aggregation operator
		to handle the 'empty case'.
	\item \texttt{isCompanyExclusive()}:
		In this method, we are provided with a specific disk (id),
		and we want to check if the disk's company matches that of all of it's
		associated RAMs. The tricky part was that we need to be able
		to check if an entier group (nested querry) matchs the same value (which was also a nested querry);
		to solve this - we used the 'ALL' quantifier - which was actually it's only use
		case for us in this project.
\end{itemize}

\subsection{Advanced API}
The interesting method in the Advanced API was \texttt{getClosePhotos()}
which not only required the longest query we have made so far,
but also required us to utilize the 'GROUP'
operator while filtering the associated groups using a nested
query that counts the number of matching photos - in order to be 
able to determine the proportion of the photos which were in the same group
as the one asscoicated with the argument \texttt{photoID}.\\
Morever - the hardest part of the design of this method (and perhaps the project)
was the requirement to handle the 'empty case' - in which 
we have to return all photos if the provided photo is not on any disk.\\
This esstially required a branching in the logic - but since
we were only allowed to use a single atomic sequence of queries -
we have created a modular query that uses the 'UNION' operator
to effevely implement an 'if' statment without actually using 'if'.