\section{ERD}
\subsection{}
The tables are:
\begin{enumerate}
    \item Policy(\underbar{Number}, \underbar{Company}, Expiration)
    \item Person(\underbar{ID}, Name)
    \item Person.Record(\underbar{ID}$^FK$, Record)
    \item Vehicle(\underbar{PlateNumber}, Manufacturer)
    \item Car(\underbar{PlateNumber}, Type)
    \item Bus(\underbar{PlateNumber}, Operator)
    \item Insurance(\underbar{PlateNumber}$^{FK}$, \underbar{Number}$^{FK}$, \underbar{Company}$^{FK}$, Agent)
    \item Drive(\underbar{ID}$^{FK}$, \underbar{PlateNumber}$^{FK}$)
    \item Accident(\underbar{front.PlateNumber}$^{FK}$, \underbar{back.PlateNumber}$^{FK}$, Location, Data)
\end{enumerate}

\subsection{}
\begin{enumerate}
    \item Possible. It would require
    that the Insurance table would not contain any entiries
    with a PlateNumber key that equals to that of the car.
    \item Possible. While a Person relates to precisely
    one Vechicle - a Vechicle (Bus) might not be related
    to any person.
    \item Possible. There is no constraint on how many
    different Persons drive the same Car, only the other way around.
\end{enumerate}

\subsection{}
The phrasing on the question suggests exactly one officer
arrives at each Accident so I will assume that is the case.\\
Additionally - no information was specified as for how
different officers are identified, hence I will assume each officer
can be fully and uniquely described with a key called 'OfficerID'.\\

Under these assumptions, the required information can be added
by appending two attributes to the Accident relation: 'OfficerID' and 'OfficerArrivalTime'.\\
After these changes, the table represeting accidents will look like:\\
Accident(\underbar{front.PlateNumber}$^{FK}$, \underbar{back.PlateNumber}$^{FK}$, Location, Data, OfficerID, OfficerArrivalTime)

\subsection{}
Alice is correct.
Since in both the solution of Bob and that of Charlie
there is a unique connection going from the front accident
participant to the back accident participant;
This means that any Vechicle which participates
in an accident as the front participant
must participate in accidents as the front participant
which \textbf{exactly one} other car
\footnote{which will be the back participant in that context}.
The problem with that is the fact that this means
accidents in which multiple cars crash to the back of a single
car (such as in the example before) cannot be described by the scheme.