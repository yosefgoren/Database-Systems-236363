\section{RA,RC}
\subsection{}
\begin{enumerate}
    \item \[
        \pi_{Name}(
            \sigma_{Company=BIG}(
                Person\Join Drive \Join Insurance
            )
        )
    \]
    Note how $Person\Join Drive$ Is uniquely identified
    by ID since a person cannot have more than one car,
    hence ID is uniquely related to some PlateNumber so far.\\

    In addition an entity of type Vehicle must relate
    to at most one entity of Policy,
    so a PlateNumber is uniquely related to some Number (from Policy).\\

    Thus in $Person\Join Drive\Join Insurance$ - each ID
    is uniquely related to a Number.
    
    \item $$\{n_{Name}\mid
        \exists \left(i_{ID}, p_{PlateNumber}, n_{Number}, a_{Agent}\right):
    $$$$
        Person(i_{ID}, n_{Name})\wedge
        Drive(i_{ID}, p_{PlateNumber})\wedge
        Insurance(p_{PlateNumber}, n_{Number}, BIG, a_{Agent})
    \}    
    $$
\end{enumerate}

\subsection{}
\subsection{}
\begin{enumerate}
    \item For readablility define:
    $$AccidentDates=\pi_{Time}(Accident)$$
    And the answer is:
    $$
        (AccidentDates\times AccidentDates)\setminus AccidentDates
    $$
\end{enumerate}

\subsection{}
\begin{enumerate}
    \item The query returns the set of agenets that work with insurance
    companies which insure every Manufacturer in the database.
    \item That would not be possible since this query
    can be reduced to the '÷' which can be further reduced
    to the $\setminus$ operator (which we know is independent).\\
\end{enumerate}

\subsection{}