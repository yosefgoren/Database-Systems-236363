\section{}
\subsection{}
$X$ is not in $BCNF$.\\
Proof.\\
From transitivity and reflexivity, $CG\rightarrow H$.\\
Note $CG$ is not a superkey in $R_3$ since $D$ is not implied
by any FD since it only appears as the left operand.\\
Thus $CG \rightarrow D$ is not satisfied and $R_3\notin BCNF \Rightarrow X\notin BCNF$.

\subsection{}
No.\\
Proof.
\begin{itemize}
	\item $e$ only appears as the right operand.
	\item $e$ is not prime.
	\item Any properkey does not contain $e$.
	\item $c$ is not a superkey in $R_2$.
	\item $c\rightarrow e$ violates the properties of $3NF$.
\end{itemize}

\subsection{}
By running the algorithm seen in class we get to:
\begin{center}
	\begin{tabular}{| c | c | c | c | c | c | c |}
		\hline
		A	&B	&C	&D	&E	&G	&H	\\
		\hline
			&b	&c	&d	&	&	&	\\
		\hline
			&	&c	&	&e	&	&h	\\
		\hline
			&	&c	&d	&	&g	&h	\\
		\hline
	\end{tabular}
	$\Longrightarrow_{C\rightarrow E}$
	\begin{tabular}{| c | c | c | c | c | c | c |}
		\hline
		A	&B	&C	&D	&E			&G	&H	\\
		\hline
			&b	&c	&d	&\textbf{e}	&	&	\\
		\hline
			&	&c	&	&e			&	&h	\\
		\hline
			&	&c	&d	&\textbf{e}	&g	&h	\\
		\hline
	\end{tabular}
\end{center}
And thus no more steps can be deduces and via the correctness of the algorithm -
the decomposition does not preserve information.

\subsection{}
Let
$$
R := \{(a_1, b, c, d, e, g_1, a_1), (a_2, b, c, d, e, g_2, a_2)\}
$$
Note that $R\models F$,
additionally $(a_2, b, c, d, e, g_1, a_2)\in R_1\bowtie R_2\bowtie R_3$,
thus the decomposition does not preserve information.


\subsection{}
First we apply the confusing algorithm seen in the tutorial:

\begin{itemize}
	\item $Z_F := \{A,H\}$
	\item On inspecting $R_1$:\\ $\{A,H\}\cup (\{A,H\}\cap (\{A,H\}\cap \{B,C,D\})^+\cap \{B,C,D\})=\{A,H\}$
	\item On inspecting $R_2$:\\ $B\notin R_2 \Rightarrow  B\notin Z_F$
	\item On inspecting $R_3$:\\ $B\notin R_3 \Rightarrow  B\notin Z_F$
\end{itemize}
Thus $B\notin Z_F\Rightarrow $ Dependencies are not preserved.