\section{Scheme Analysis}
\subsection{}
The keys:
\begin{itemize}
	\item $K_1:= EG$.\\
	By applying the algorithm we see:\\
	Note that $A$ can be derrived from $BCDEG$ by the FD $G\rightarrow ABCD$
	and the Decomposition property.\\
	By applying the algorithm for finding keys 3 more times,
	we get rid of $BCD$. And we are left with $EG$.\\
	Note that $E\not\rightarrow G$ and $G\not\rightarrow E$ - 
	hence $K_1$ is a proper-key.

	\item $K_2:= ABCD$.\\
	By the rule $BD\rightarrow G$ we get rid of $G$,
	and by the rule $AC\rightarrow D$ we get rid of $D$.\\
	Thus we are left with $K_2$ while not more attributes can be removed -
	so by the correctness of the algorithm - $K_2$ is a proper-key.
\end{itemize}
\subsection{}

\subsection{}
The claim is correct.\\
\textbf{Proof.}\\
Let $K:=A_{i_1} A_{i_2} ... A_{i_k}$ s.t. $K$ is the left operand in $f$,\\
and so $\forall j\in\{1... k\}: i_j\in \{1...n\}$.\\

Now we show that $K$ is a super-key:\\
Let $j\in\{1...n\}: A_j\notin K$.\\
$A_j$ must be the right operand of $f$,
Thus by Decomposition be can derive that $K\rightarrow A_j$.
Thus $K^+=R$.\\
Thus $F$ satisfies the BCNF property and from a theorem from the lectures,
this sufficient to show that $(R,F)\in BCNF$.

\subsection{}
The claim is incorrect.\\
\textbf{Example.}\\
Let $F:=A_1\rightarrow A_2A_3$.\\
Each $A_i$ appears exactly once. But by the definition of a minimal cover - 
on the right side of each operand in the cover - there must be exactly one
attribute.