\section{Scheme Analysis}
\subsection{}
The key is $K := ADEG$.\\
Now we show for each attribute in $R$ that it can be derrived from $K$.
\begin{itemize}
	\item $H$: 
	\begin{itemize}
		\item $G\rightarrow H$ (from $F$)
		\item $K\rightarrow KH$ (Augmentation)
		\item $KH\rightarrow H$ (Reflexivity)
		\item $K\rightarrow H$ (Transitivity)
	\end{itemize}
	\item $B$: 
	\begin{itemize}
		\item $G\rightarrow B$ (from $F$)
		\item $K\rightarrow KB$ (Augmentation)
		\item $KB\rightarrow B$ (Reflexivity)
		\item $K\rightarrow B$ (Transitivity)
	\end{itemize}
	\item $C$: 
	\begin{itemize}
		\item $E\rightarrow CH$ (from $F$)
		\item $CH\rightarrow C$ (Reflexivity)
		\item $E\rightarrow C$ (from $F$)
		\item $K\rightarrow KC$ (Augmentation)
		\item $KC\rightarrow C$ (Reflexivity)
		\item $K\rightarrow C$ (Transitivity)
	\end{itemize}
\end{itemize}

\subsection{}
\begin{enumerate}
	\item
	The claim is correct.\\
		Proof.\\
		\textbf{Direction '$\Rightarrow$'}
		\begin{enumerate}
			\item $B\rightarrow G$ (from $F$)
			\item $BG\rightarrow G$ (Augmentation)
			\item $BG\rightarrow E$ (from $F$)
			\item $G\rightarrow H$ (from $F$)
			\item $BG\rightarrow BH$ (Augmentation)
			\item $BH\rightarrow H$ (Reflexivity)
			\item $BG\rightarrow H$ (e, f, Transitivity)
			\item $BH\rightarrow EGH$ (b,c,g, Union)
		\end{enumerate}
		\textbf{Direction '$\Leftarrow$'}
		Let $A_i\notin F$.\\
		$\Rightarrow$ $A_i$ cannot be derrived from the left operand of $f$\\
		$\Rightarrow$ The left operand of $f$ is not a super-key\\
		$\Rightarrow$ $(R,F)\notin BCNF$
	\item
	The claim is incorrect.\\
		\textbf{Example.}\\
		Let $R := \{(a_1, b_1, c_1, d_1, e_1, g_1, h_1), (a_2, b_2, c_1, d_1, e_2, g_2, h_1)\}$.\\
		Note that $R\models F$, while $R\not\models (CH\rightarrow AE)$ thus,
		from the soundness theorem - we can deduce there is no proof.
\end{enumerate}

\subsection{}
\begin{center}
	\begin{tabular}{| c | c | c | c | c | c | c |}
		\hline
		A	&B	&C	&D	&E	&G	&H	\\\hline
		a	&b	&	&	&	&g	&	\\\hline
			&b	&c	&	&e	&	&	\\\hline
			&	&	&d	&e	&	&h	\\\hline
	\end{tabular}
	$\Longrightarrow_{B\rightarrow G}$
	\begin{tabular}{| c | c | c | c | c | c | c |}
		\hline
		A	&B	&C	&D	&E	&G	&H	\\\hline
		a	&b	&	&	&	&g	&	\\\hline
			&b	&c	&	&e	&g	&	\\\hline
			&	&	&d	&e	&	&h	\\\hline
	\end{tabular}
	$\Longrightarrow_{BG\rightarrow E}$
	\begin{tabular}{| c | c | c | c | c | c | c |}
		\hline
		A	&B	&C	&D	&E	&G	&H	\\\hline
		a	&b	&	&	&e	&g	&	\\\hline
			&b	&c	&	&e	&g	&	\\\hline
			&	&	&d	&e	&	&h	\\\hline
	\end{tabular}
	$\Longrightarrow_{E\rightarrow CH}$
	\begin{tabular}{| c | c | c | c | c | c | c |}
		\hline
		A	&B	&C	&D	&E	&G	&H	\\\hline
		a	&b	&c	&	&e	&g	&h	\\\hline
			&b	&c	&	&e	&g	&h	\\\hline
			&	&c	&d	&e	&	&h	\\\hline
	\end{tabular}
	$\Longrightarrow_{C\rightarrow D}$
	\begin{tabular}{| c | c | c | c | c | c | c |}
		\hline
		A	&B	&C	&D	&E	&G	&H	\\\hline
		a	&b	&c	&d	&e	&g	&h	\\\hline
			&b	&c	&d	&e	&g	&h	\\\hline
			&	&c	&d	&e	&	&h	\\\hline
	\end{tabular}
\end{center}

\subsection{}
\begin{center}
\begin{tabular}{| c | c | c |}
	\hline
			&$X_2$	&$X_3$	\\\hline
	BCNF	&\cmark	&\xmark	\\\hline
	3NF		&\cmark	&\xmark	\\\hline
\end{tabular}
\end{center}

Both \xmark 's can be demostrated by $C\rightarrow D$.\\
